\documentclass[12pt,a4paper,oneside]{article}
\usepackage[dutch]{babel}
\usepackage{graphicx}
\usepackage{float}
\usepackage{alltt}
\usepackage{layout}
\usepackage[justification=centering]{caption}

\begin{document}

\title{Internettechnologie \\ Practicum 2: Reflectie}
\author{Titouan Vervack \& Caroline De Brouwer}
\maketitle

\thispagestyle{empty}
Native apps voor elk toestel of toch maar kiezen voor cross-platform met web-based apps? Beide, zeggen developpers. Uit onderzoek blijkt dat men meer en meer elk project individueel bekijkt en op basis daarvan kiest wat het beste past. \\\\
Natuurlijk zijn er voordelen en nadelen verbonden aan elke aanpak. Gebruikers zijn het meest vertrouwd met native apps, omwille van het gemak van de centrale app-store. En hoewel er al veel hulpmiddelen bestaan voor HTML5, is een native app nog steeds iets performanter en heeft betere toegang tot de hardwarecomponenten (zoals camera) en gegevens (zoals contactpersonen) op het toestel. Daarbij heeft een HTML5-app meer toegang tot het internet nodig, en is daarom minder veilig en heeft een beperkte offline toegankelijkheid.\\\\
Daartegenover staat dat HTML5 makkelijker is in development; er moet slechts \'e\'en app gemaakt worden die je kan gebruiken op alle toestellen. Daarbij is HTML ook een vertrouwde taal voor min of meer alle programmeurs, terwijl bijvoorbeeld Objective-C voor iOS veel minder gekend is. Dit zal de kost van het ontwikkelen doen dalen en de snelheid ervan opdrijven. Ook moeten HTML-5 ontwikkelaars, in tegenstelling tot native app ontwikkelaars, zich geen zorgen maken over toestemming van Apple of Google en moeten ze ook geen stuk van hun winst delen.\\\\
Kortom, er moet misschien nog een beetje gesleuteld worden aan HTML5, maar het is vooral aan de gebruiker om vertrouwd te worden met webgebaseerde apps. Voor nu zal de keuze afhangen van welke middelen je voorhanden hebt en welke functionaliteit je wilt om al dan niet te kiezen voor een platformspecifieke app.

\end{document}