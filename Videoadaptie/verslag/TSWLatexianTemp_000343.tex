\documentclass[12pt,a4paper,oneside]{article}
\usepackage[dutch]{babel}
\usepackage{tikz}
\usepackage{a4wide}
\usepackage{amsmath}
\usepackage{amssymb}
\usepackage{rotating}
\usepackage{listings}
\usepackage{float}
\usepackage{color}
\usepackage{fancyhdr}
\definecolor{dkgreen}{rgb}{0,0.6,0}
\definecolor{gray}{rgb}{0.5,0.5,0.5}
\definecolor{mauve}{rgb}{0.58,0,0.82}
 
\lstset{ %
  language=PYTHON,                
  basicstyle=\footnotesize,           
  numbers=left,                  
  numberstyle=\tiny\color{gray},  
  stepnumber=2,                        
  numbersep=5pt,                  
  backgroundcolor=\color{white},      
  showspaces=false,               
  showstringspaces=false,        
  showtabs=false,                 
  frame=single,                   
  rulecolor=\color{black},        
  tabsize=2,                      
  captionpos=b,                   
  breaklines=true,                
  breakatwhitespace=false,        
  title=\lstname,                                                  
  keywordstyle=\color{blue},          
  commentstyle=\color{dkgreen},       
  stringstyle=\color{mauve},         
  escapeinside={\%*}{*)},            
  morekeywords={*,...},              
  deletekeywords={...}              
}
\usepackage{titling}

\setlength{\droptitle}{-10em} 
\begin{document}

\title{Videoadaptatie \& schaalbare videocodering}
\author{Stefaan Vermassen \\
	     Titouan Vervack}
\date{\today}
\pagestyle{fancy}
\fancyhf{}
\fancyhead[R]{Stefaan Vermassen \& Titouan Vervack}
\fancyhead[L]{Multimedia: practicum 4}
\fancyfoot[C]{\thepage}
\maketitle
\section{Kwaliteitsschaalbaarheid}
Voor de single-layer codering kunnen we de benodigde waarden voor het opstellen van de RD-curve aflezen uit de output van de \verb$H264AVCEncoderLibTestStatic.exe -pf main.cfg$. We laten de QP waarde vari\"eren in \verb$layer0.cfg$. Om de vergelijking met SVC te laten opgaan, vergelijken we een simulcast van single layers met SVC. We plotten de bekomen PSNR waarden tegenover bitrates [X,X+Y,X+Y+Z].
Om de waarden voor de kwaliteitsschaalbaarheid met SVC te bepalen, extraheren we de layers met het commando.
\begin{verbatim}
BitStreamExtractorStatic.exe output.264 extracted.264 -f{0,1,2}
\end{verbatim}
We decoderen de 3 bekomen bitstromen en meten de PSNR-Y.
\begin{verbatim}
H264AVCDecoderLibTestStatic.exe extracted.264 output.yuv
VQMT BBB_640x360_24.yuv output.yuv 640 360 50 data PSNR
\end{verbatim}
\begin{figure}[H]
  \begin{center}
    \input{1.tex}
    \caption{Configuraties kwaliteitsschaalbaarheid}
    \label{graph:graph1}
  \end{center}
\end{figure}
We merken op dat SVC met inter-layer predictie duidelijk veel beter presteert dan een simulcast van single layers of een SVC zonder inter-layer predictie. Bij een multicast van single layers zal een betere kwaliteit single-layer opnieuw de basisinfo bevatten die ook in een lagere kwaliteit single-layer zit. Zo wordt er dus veel info dubbel opgeslaan. De multicast van single layers en de SVC zonder inter-layerpredictie zijn ongeveer even effici\"ent.
\section{Spatiale schaalbaarheid}
We maakten eerst een geschaalde versie van \verb$BBB_640x360_24.yuv$ aan met het commando
\begin{verbatim}
DownconvertStatic.exe 640 360 BBB_640x360_24.yuv 320 180 BBB_320x180_12.yuv
\end{verbatim}
De PSNR waarden werden berekend zoals besproken in de vorige opgave.
\begin{figure}[H]
  \begin{center}
    \input{2.tex}
    \caption{Configuraties spatiale schaalbaarheid}
    \label{graph:graph1}
  \end{center}
\end{figure}
\noindent Als we kijken naar de punten van de single-layer, zien we dat die veel beter presteert dan de SVC. Dit komt omdat single-layer het minst informatie bevat. We vinden het raar dat SVC zonder inter-layer predictie beter presteert dan met.
\end{document}